% Este archivo es parte de la memoria del proyecto fin de carrera
% de Aarón Bueno Villares. Protegida bajo la licencia GFDL.
% Para más información, la licencia completa viene incluida en el
% fichero fdl-1.3.tex

% Copyright (C) 2010 Aarón Bueno Villares

\section{Introducción}
\label{sec:introduccion}

\fpt es \----o pretende ser\---- una herramienta
divulgativa para que estudiantes e interesados en asuntos de biología
evolutiva puedan aprender, la evolución de la vida en la Tierra, tanto
a macroescala como focalizándose en cualquier grupo concreto de organismos
(alguna familia bacteriana, felinos, delfines, hominidos, etcétera).

El proyecto nace en respuesta a varias causas:
\begin{itemize}
\item Ausencia de softwares similares.
\item Deseo de poner en práctica patrones de diseño.
\item Inspiración en el software \textit{gource}.
\item Ganas de aprender a programar bajo librerías 3D.
\item Proyecto de «entrenamiento» para mi posterior proyecto fin de
  carrera, que será un visualizador de algoritmos y modelos de
  computación, en dónde solo trabajaré con primitivas gráficas para
  realizar animaciones, al igual que en \textit{gource} y en
  \textit{FreePhyloTree}.
\end{itemize}

Por todos éstos motivos, decidí inscribirme en el CUSL V, y así tener
un incentivo para poder poner este proyecto en marcha.


